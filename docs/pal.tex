Title: Finding the Shortest Possible Palindromic Pangram
Subtitle: An Admissable Heuristic and a Method for its Computation

Abstract

ITA Software lists several puzzles on its website that it uses to
filter out candidates during its hiring process. One of those puzzles
asks the person to find the shortest possible sequence of words that
is both a palindrome and a pangram, using a supplied word list. This 
paper describes an admissable heuristic to use in searching for a
solution to this puzzle, and a reasonably efficient method for its 
computation.

Background

This is a paper that describes my solution to the palindromic pangram
problem proposed by ITA software.

The problem statement is fairly straightforward. Given a list of
words, candidates are to find the shortest sequence of a subset
of those words that is both a palindrome and a pangram. For my
solution, I take the assumption that shortest means in total number of
total characters, excluding spaces, as opposed to total number of
words.

Search Problem Definition

In order to be able to use general search algorithms, the problem
needed to be restated as a search problem. My interpretation of this
problem as a search problem.... I describe a state in the search space
as a list of words on the left and right side of the candidate word
sequence built thus far, along with the remaining characters in order
to complete a full pangram. The initial state would thus be one in
which there are no words on either the left or right side, and all 26
letters of the alphabet remain to be found.

The successor states from a given state can be found by adding any
word that does not violate the palindrome constraint to the shorter of
the two sides. If the left side is longer, we can take the letters on
the left side that are unmatched on the right side as a string,
reverse it, and then find all the words that end with that string. If
the right side is longer, we again reverse the unmatched string, and
find all the words that start with that string. For each such word
that we find, we create a new successor state by adding that word to
the shorter side. As we go along, we keep track of the letters that
are remaining to be found.

The goal state can be defined as a state in which the remainder of the
letters to be found is itself a palindrome, and no letters remain to
be found.

The cost of a path to any such given state is simply the number of
characters in all the words that have been visited.

My solution is based around a straightforward search approach as
described by Norvig and Russell in AI: A Modern Approach.

I experimented with several of the algorithms advised in the book with
varying degrees of success.

The first noteworthy aspect of the program is the treatment of the
palindrome building process. I was inspired by Norvigs approach to
generating the longest possible palindrome (1) and found it useful to
view the search as alternating between the left and right sides of the
palindrome. That is, starting with an empty set of words, I add any
word on the left to yeild the next set of search nodes. Then, from any
given such state, the only legal successor states are words having the
reverse of the remaining characters as either making up the whole
word, or as making up the first portion of the word.

The next noteworthy piece is the pangram tracking.

The final noteworthy piece is the heuristic function. After describing
the problem in a manner compatible with the general search algorithsm,
I found through experiementation and subsequently through calculations
that a good heuristic was the only way to make this problem
approachable. The only heuristic I found at first that had some degree
of success was based on an assumption that the letters that had the
fewest words available would require more characters in order to
complete the solution. I defined the function specifically as the sum
of the squares of the rankings of the remaining characters. Using this
approach, I am able to find reasonably short solutions within 1000 or
so nodes. This heuristic, while useful and fast at finding a solution
efficiently, does not find an optimal solution, because we have no way
of proving that just because its not likely that there is a J word for
instance having that letter, that there istn' a perfect word out there
having that letter combintaion.

The most obvious heuristic is to use the number of letters remaining *
2 - 1. This is the absolute minimum that could be found that would
meet the requirements. It is good because it never overestimates the
remaining letters. It is bad in the sense that it doesn't prune enough
of the tree to really get the problem within a solvable time. I
estimate that the time required to solve this problem with even the
best algorithms to be many thousands of years on my machine.

The best heuristic I could think of that would be both optimal and
prune the tree enough is itself difficult to compute. Basically, it
relaxes the problem constraint and uses the answer of that as the
basis for the remaining path cost. It is the required number of
characters to form any pangram given the set of remaining characters *
2 - 1. This is admissable because it is based on a subconstraint of
the original problem. It ends up yielding answers similar to the fast
heuristic described earlier.

The difficulty comes in trying to compute the heuristic. I started by
trying to use the same approach I was taking for the larger problem,
but ran into the same difficulty of the search tree being too
complex. Then I realized I could scale the definition of the problem
back even further given that I only had the requirement of finding the
cost of the path to the shortest pangram, rather than the actual
pangram itself.

Also, it was around this time that I realized I could treat the
letters required as bits of an integer. Given that the most characters
it should take to form any required letter set, even a full pangram,
in around 60 characters, this means that I could solve all of the path
costs in only 2 ^ 26 bytes of memory. Thus I could hold cache the
h-cost completely in memory. Further, I found a reasonable algorithm
for computing this entire table in a reasonable amount of time.


Useful Statistics

Total # of words in set:
Letter distribution (# of words having the letter): a - 123, b 1235, etc...
Letter distribution (total # of letters): a - 2384, b 3294, etc...



References

AI: A Modern Approach
On Lisp
Common Lisp The Language, Second Edition
Norvig Palindrome Page
ITA Software Website Problem Description
